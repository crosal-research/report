% Created 2017-04-07 Sex 17:46
% Intended LaTeX compiler: pdflatex
\documentclass{report}
               \pagestyle{fancy}
\usepackage[utf8]{inputenc}
\usepackage[T1]{fontenc}
\usepackage{graphicx}
\usepackage{grffile}
\usepackage{longtable}
\usepackage{wrapfig}
\usepackage{rotating}
\usepackage[normalem]{ulem}
\usepackage{amsmath}
\usepackage{textcomp}
\usepackage{amssymb}
\usepackage{capt-of}
\usepackage{hyperref}
\usepackage{paralist}
\usepackage{tcolorbox}
\usepackage[table]{xcolor}
\usepackage{lipsum}
\usepackage{caption}
\usepackage{tabu}
\usepackage[subpreambles=true]{standalone}
\usepackage{import}
\usepackage{setspace}
\usepackage{graphics}
\usepackage[linktocpage=true]{hyperref}
\usepackage{tocloft}
\usepackage{minitoc}
\usepackage[portuguese, english]{babel}
\usepackage[utf8]{inputenc}
\usepackage{subfig}
\date{}
\title{}
\hypersetup{
 pdfauthor={user},
 pdftitle={},
 pdfkeywords={},
 pdfsubject={},
 pdfcreator={Emacs 25.1.1 (Org mode 9.0.5)},
 pdflang={English}}
\begin{document}

\thispagestyle{firstpagestyle}
\IssueTitle{Single Digit Selic in 1H2017?}
\NewsAuthor{João Mauricio Rosal, \small\it Chief Economist, PhD}
\NewsEmail{joao.rosal@bgcpartners.com}
\JournalIssue
    \begin{tcolorbox}[colbak=red!5!white, colframe=red!0!white]
      \NewsItem{Single Digit Selic in 1H2017?}
      \begin{compactitem}
      \item \textit{The huge budget for rate cuts and limited risks to the easing cycle calls for stronger action by BCB.}
      \item \textit{Accordingly, we believe the Central Bank will overcome its excessive conservatism this time around and opt for 125pbs cut.}
      \item \textit{Moreover, it may also signal that single digit rates may be observed already in 1S2017, entailing a possible cut of 125bps in May's meeting as well.}
      \end{compactitem}
    \end{tcolorbox}
\vspace{-0.5cm}


\section{Our COPOM Call}
\label{sec:orgc2ab3ef}
\begin{compactitem}[$\diamond$]
\item \textbf{Our Call}: We believe the Central Bank (CB) shall cut the Selic by
125bps this coming Wednesday. As a matter of fact, we understand it
could dare further, however, based upon the conservatism shown till
the moment, we feel more comfortable in expecting a more contrived
move this time around.

In addition, it is not unlikely the CB hints that it may pursue a
single digit rate already in 1H2017. In this respect, any move shall
depend on the CB's risk assessment with regards to short-term risks,
in particular, to the outcome of social security reform, which
revolves around subjective judgments as well as information that
only those in Brasilia may hold.

\item \textbf{The Size of the Budget}. The incoming data continued to underscore
that the economy is stuck in a depression and that signs of recovery
are mixed, at best. As matter of example, the Central bank's GDP
tracker shrunk 0.26\%momsa in January, while industrial production
tanked 0.7\%mom in February. To put it mildly, this is rather
disappointing for an economy with an output gap estimated at about
3-4\% below potential.

On the inflation front, CPI inflation is already on target, having
reached 4.57\% in March. Moreover, the average month-on-month
annualized rate in the 1Q2017 has reached a dismal 3.8\%, while the
2018E expected inflation is at 4.5\%. Hence, if one sums all that up
with the state of the economy, it is not difficult to see that any
type of taylor rule should advocate a rate at about 200bps below
neutral, currently standing at 9.6\% by market metrics (real rate of
5.1\% plus inflation target).

Therefore, when we take the aforementioned figures into
consideration, along with the fact that the fiscal policy is on the
contraction mode and subsidized credit has collapsed, we understand
there is still a huge budget of cuts ahead of us, of about 450bps.

\item \textbf{Risks}. As usual, risks abound, but as far as the intervening
period between this week's meeting and the one in May is concerned,
we reckon two of them call for special attention. On the local
front, a negative outcome for social security reform may lead to a
meaningful market dislocation, particularly in face of the fact that
local assets look a bit pricey relative to the uncertainties
concerning the reform.

On the global camp, besides the usual worries concerning
international financial conditions, the first round of French
Presidential election scheduled to April 23 can potentially send
waves of shocks across the globe. However, from our vantage point,
their possible impact seem less worrying relative to those on the
local front.

\item \textbf{Net/Net}: Be all that as it may, while the realization of the
aforementioned negative outcomes may indeed shorten the length of
the easing cycle, we understand that a 125bps cut in this
Wednesday's meeting wouldn't lead the Central Bank into any radical
u-turn, should any of these events occur. In the meantime, on the
margin, the move could entail a more expedient recovery of the
economy, which has already taken far too long, with many negative
feedback loops, in particular, over fiscal policy outlook.
\end{compactitem}


\newpage

\section{Forecast}
\label{sec:org4ff6bb4}

\rowcolors{2}{grey!15}{white}
\begin{center}
\begin{tabular}{lrrr}
\textbf{Variable} & \textbf{2016} & \textbf{2017E} & \textbf{2018E}\\
\hline
\hline
GDP (yoy \%) & -3.5 & 0.3 & 2.5\\
Selic Rate (\%) & 13.25 & 8.00 & 7.5\\
Fx Rate (BRL per USD) & 3.20 & 3.25 & 3.30\\
CPI (yoy \%) & 6.3 & 4.0 & 4.3\\
Gross Debt to GDP (\%) & 73.0 & 77.2 & 83.2\\
Primary Surplus (\% GDP) & -2.1 & -2.0 & -1.0\\
Trade Balance (\% GDP) & 2.0 & 1.5 & 1.5\\
Current Account (\%GDP) & -1.2 & -1.5 & -2.0\\
Foreign Reserves (xUSD Gross Liabilities) & 1.1 & 1.1 & 1.1\\
\hline
\end{tabular}
\end{center}
\newpage

p
\section{Disclaimer}
\label{sec:orgce20997}
\url{http://www.bgcpartners.com} \\
\textbf{CONFIDENTIAL:} This document has been sent to you by one of
the BGC entities (collectively BGC) Please see important legal
information and disclaimer relating to this mail at the following
links: \url{http://www.bgcpartners.com/disclaimers/}

Please see for BGC Disclosures. The link contains company and FCA
registration numbers. This e-mail, including its contents and
attachments, if any, are confidential. If you are not the named
recipient please notify the sender and immediately delete it. You may
not disseminate, distribute, or forward this e-mail message or
disclose its contents to anybody else. Copyright and any other
intellectual property rights in its contents are the sole property of
BGC and its affiliates. E-mail transmission cannot be guaranteed to be
secure or error-free.  The sender therefore does not accept liability
for any errors or omissions in the contents of this message which
arise as a result of e-mail transmission.  If verification is required
please request a hard-copy version. Although we routinely screen for
viruses, addressees should check this e-mail and any attachments for
viruses. We make no representation or warranty as to the absence of
viruses in this e-mail or any attachments. Please note that to ensure
regulatory compliance and for the protection of our customers and
business, we may monitor and read e-mails sent to and from our
server(s).  The registered offices of the BGC entities are at 1
Churchill Place, London, E14 5RD.  For any issues arising from this
email please reply to the sender.  The FCA register appears at
\url{http://www.FCA.org.uk/register/}.  The FCA regulates the financial
services industry in the United Kingdom and is located at 25 The North
Colonnade, Canary Wharf, London, E14 5HS.  BGC Financial LP CFTC Rule
1.55(K) Firm Specific Disclosure Statement
\end{document}
